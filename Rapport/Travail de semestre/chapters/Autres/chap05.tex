%
% File: chap01.tex
% Author: Zeller Quentin
% Description: Introduction
%
%\let\textcircled=\pgftextcircled
\chapter{Produits annexes}
\label{chap:marche}



%=======

\section{HTML to JPEG}
\label{subsec:subsec05b}
Dans l'id�e de construire des overlays de qualit�, ainsi que d'utiliser ce qui est d�j� existant dans le monde du web, il peut �tre int�ressant de consid�rer la solution HTML pour la cr�ation d'overlay.
Par exemple une solution populaire pour cette fonction est Imgkit\footnote{https://pypi.python.org/pypi/imgkit/0.1.1} qui est un wrapper de Wkhtmltopdf utilisant le moteur de rendu Webkit, utilis� notamment pour Chrome, Opera et Safari. Beaucoup d'�l�ments du moteur de rendu sont param�trables, les dimensions, les polices, etc...

Ces librairies permettent de transformer des pages Web en fichiers images ou PDF. L'�norme avantage �tant l'aspect "responsive" des pages web. L'affichage dans diff�rents rapport, 16/9, 4/3, etc... Sont grandement facilit�. Il serait �galement possible d'utiliser cet overlay dans un autre cas de figure et notamment HbbTV.

\section{Emulateurs Hbbtv}
Il existe d'autres emulateurs test pour HbbTV. Certains sont moins lourd et peuvent �tre pratiques pour des petits test mais son moins fiable �tant donn�e qu'ils ne sont que des simulateurs.
Notamment : FireHbbTV est un module firefox permettant de tester les applications au sein m�me du navigateur Firefox. Il utilise cependant le moteur de rendu Firefox qui n'est pas forc�ment le m�me que sur les t�l�vision et en tout cas pas le m�me qu'op�ra qui utilise webkit.


%=========================================================