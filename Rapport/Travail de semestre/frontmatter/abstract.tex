%
% File: abstract.tex
% Author: V?ctor Bre?a-Medina
% Description: Contains the text for thesis abstract
%
% UoB guidelines:
%
% Each copy must include an abstract or summary of the dissertation in not
% more than 300 words, on one side of A4, which should be single-spaced in a
% font size in the range 10 to 12. If the dissertation is in a language other
% than English, an abstract in that language and an abstract in English must
% be included.

\chapter*{Abstract}
\begin{SingleSpace}
\initial{C}e travail consiste en une �tude de march� concernant l'int�gration de contenu dans des flux vid�o en direct. Il posera les bases pour l'�laboration du projet en soi. C'est a dire, le d�veloppement d'une solution permettant la r�cup�ration et l'affichage de donn�es dans un flux vid�o en temps r�el. Cette application doit pouvoir �tre utilis�e facilement, sans connaissance ni infrastructures particuli�res. Elle est en somme, une application tout publique. Cette recherche a comme ambition de rendre l'affichage de donn�es sur une vid�o live plus ais�, les solutions actuelles �tant restreintes � des cas particuliers.
\\

\begin{description}
	\item[\nameref{chap:Introduction}] Une �tude du march� succincte o� les solutions seront discut�es bri�vement � propos de leurs p�n�trations sur le march�, leurs prix, ainsi que les fonctionnalit�s qu'elles proposent. (\autoref{chap:Introduction})
	\item[\nameref{chap:OpenCV}] Discussion d'OpenCV, une librairie pour le traitement multim�dia tr�s r�pandue. Elle permet l'analyse et la retouche des vid�os et est connue en particulier pour ses fonctions de machine learning. (\autoref{chap:OpenCV} )
	\item[\nameref{chap:HBBTV}] Discussion de la solution applicative destin�e aux postes de t�l�vision europ�ens. HBBTV est une solution permettant l'ajout de contenu et l'interaction directe avec les utilisateurs de la t�l�vision conventionnelle que se soit sur le c�ble ou via la t�l�vision IP. (\autoref{chap:HBBTV})
	\item[\nameref{chap:Wowza}] Discussion � propos du serveur de contenu vid�o Wowza; Un service permettant la transcription et la distribution de flux vid�o direct ou a la demande. (\autoref{chap:Wowza})
	\item[\nameref{chap:marche}] Discussion des autres solutions disponibles sur le march�. (\autoref{chap:marche})
	
\end{description}










\end{SingleSpace}
\clearpage